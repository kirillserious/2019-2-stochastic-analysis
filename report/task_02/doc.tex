\section{Задание №2}

\begin{enumerate}
        \item Построить датчик сингулярного распределения, имеющий в качестве функции распределения канторову лестницу. С помощью критерия Колмогорова убедиться в корректности работы датчика.
        \item Для канторовых случайных величин проверить свойство симметричности относительно $\frac12$ ($X$ и $(1 - X)$ распределены одинаково) и самоподобия относительно деления на $3$ (условное распределение $Y$ при условии $Y\in[0,\,\frac13]$ совпадает с распределением $\frac{Y}{3}$) с помощью критерия Смирнова.
        \item Вычислить значение математического ожидания и дисперсии с эмпирическими для разного объема выборок. Проиллюстрировать сходимость.
\end{enumerate}


\subsection{Задача №1}

\begin{definition}
        Пусть дано вероятностное пространство $(\R,\, \mathcal{F},\,\p)$, и на нем определена случайная величина $X$ с распределением $\p_X$. Тогда \textit{функцией распределения} случайной величины $X$ называется функция $F_X:\,\R\to[0,\,1]$, задаваемая формулой:
        $$
                F_X(x) = \p(X\leqslant x) \equiv \p_X(\,(-\infty,\,x]\,). 
        $$
\end{definition}
\begin{definition}
        Функция распределения некоторой случайной величины называется \textit{сингулярной}, если она непрерывна и ее множество точек роста имеет нулевую меру Лебега.
\end{definition}

\begin{definition}
        Из единичного отрезка $C_0 = [0,\,1]$ удалим интервал $\left(\frac13,\,\frac23\right)$. Оставшееся множество обозначим за $C_1$. Множество $C_1 = \left[0,\,\frac13\right]\cup\left[\frac23,\,1\right]$ состоит из двух отрезков; удалим теперь из каждого отрезка его среднюю часть, и оставшееся множество обозначим за $C_2$. Повторив данную процедуру, то есть удаляя средние трети у всех четырех отрезков, получим $C_3$. Дальше таким же образом получаем последовательность замкнутых множеств $C_0 \supset C_1 \supset C_2 \supset \ldots \supset C_i \supset \ldots$. Пересечение
        $$
                C = \bigcap_{i=0}^{\infty} C_i
        $$
        называется \textit{канторовым множеством}.
\end{definition}
\begin{remark}
        Канторово множество так же можно определить как множество всех чисел от нуля до единицы, которые можно представить в троичной записи при помощи только нулей и двоек. То есть
        $$
                C = \{\,0,\alpha_1\,\alpha_2\,\ldots\,\alpha_i\,\ldots\,_3\:|\:\alpha_i=0,\,2\,\}.
        $$
\end{remark}
\begin{assertion}
        Канторово множество имеет нулевую меру Лебега. \cite{shiryaev}
\end{assertion}

С помощью построенного ранее (см. раздел~\ref{task_01}) генератора схемы Бернулли смоделируем случайную величину $X$, принимающую с вероятностью $1$ значения из множества~C.
$$
        X = \sum_{k = 1}^{\infty}\frac{2\xi_k}{3^k},
        \qquad
        \mbox{где $\xi_k\sim\mbox{Bern}\left(\frac12\right)$.}
$$

В программной реализации будем рассматривать частичные суммы, для этого этого введем погрешность $\varepsilon$ и найдем такое число $n$, при котором частичная сумма будет отличаться от бесконечной не более, чем на заданную погрешность.
$$
        \sum_{k=n}^{\infty} \frac{2\xi_k}{3^k} \leqslant 2\sum_{k=n}^{\infty}\frac{1}{3^k} = \frac{1}{3^{n-1}} \leqslant \varepsilon,
$$
$$
        \Downarrow
$$
$$
        n \geqslant 1 - \left\lceil\,\log_3 \varepsilon\,\right\rceil.
$$
\begin{remark}
        Из выведенной формулы также видно, что для такой маленькой погрешности как $\varepsilon = 10^{-9}$ достаточно использовать всего $n = 20$ первых членов ряда.
\end{remark}