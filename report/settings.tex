\documentclass[a4paper, 11pt]{article}

% Внешние пакеты
\usepackage{amsmath}
\usepackage{amssymb}
\usepackage{hyperref}
\usepackage{url}
\usepackage{a4wide}
\usepackage[utf8]{inputenc}
\usepackage[main = russian, english]{babel}
\usepackage[pdftex]{graphicx}
\usepackage{float}
\usepackage{subcaption}
\usepackage{indentfirst}
\usepackage{amsthm}                  % Красивый внешний вид теорем, определений и доказательств
% \usepackage[integrals]{wasysym}    % Делает интегралы прямыми, но некрасивыми
% \usepackage{bigints}               % Позволяет делать большущие интегралы

% Красивый внешний вид теорем, определений и доказательств
\newtheoremstyle{def}
        {\topsep}
        {\topsep}
        {\normalfont}
        {\parindent}
        {\bfseries}
        {.}
        {.5em}
        {}
\theoremstyle{def}
\newtheorem{definition}{Определение}[section]

\newtheoremstyle{th}
        {\topsep}
        {\topsep}
        {\itshape}
        {\parindent}
        {\bfseries}
        {.}
        {.5em}
        {}
\theoremstyle{th}
\newtheorem{theorem}{Теорема}
\newtheorem{lemma}{Лемма}
\newtheorem{assertion}{Утверждение}[section]

\newtheoremstyle{rem}
        {0.5\topsep}
        {0.5\topsep}
        {\normalfont}
        {\parindent}
        {\itshape}
        {.}
        {.5em}
        {}
\theoremstyle{rem}
\newtheorem{remark}{Замечание}[section]

% Новое доказательство
\renewenvironment{proof}{\parД о к а з а т е л ь с т в о.}{\hfill$\blacksquare$}

% Переопределение математических штук
\newcommand{\R}{\mathbb{R}}
\newcommand{\p}{\mathbb{P}}
\newcommand{\I}{\mathbb{I}}
\newcommand{\N}{\mathbb{N}}
\newcommand{\T}{\mathrm{T}}
\newcommand{\X}{\mathcal{X}}
\DeclareMathOperator{\sgn}{sgn}
\DeclareMathOperator{\const}{const}
