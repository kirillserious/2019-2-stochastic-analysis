\section{Задание №3}

\begin{enumerate}
        \item Построить датчик экспоненциального распределения. 
        Проверить для данного распределения свойство отсутствия памяти. Пусть $X_1,\,X_2,\,\ldots,\,X_n$ независимо распределенные случайные величины с параметрами $\lambda_1,\,\lambda_2,\,\ldots,\,\lambda_n$ соответственно. 
        Найти распределение случайной величины $Y = \min\{\,X_1,\,X_2,\,\ldots,\,X_n\,\}$.
        \item На основе датчика экспоненциального распределения построить датчик пуассоновского распределения.
        \item Построить датчик пуассоновского распределения как предел биномеального распределения.
        С помощью критерия хи-квадрат Пирсона убедиться, что получен датчик распределения Пуассона.
        \item Построить датчик стандартного распределения методом моделирования случайных величин парами с переходом в полярные координаты. Проверить при помощи t-критерия Стьюдента равенство математических ожиданий, а при помози критерия Фишера --- равенство дисперсий.
\end{enumerate}

\subsection{Задача №1}

\begin{definition}
        Случайная величина $X$ имеет экспоненциальное распределение с параметром $\lambda > 0$, если ее функция распределения имеет вид
$$
        F_X(x) = 
        \begin{cases}
                1 - e^{-\lambda x},& \mbox{при $x \geqslant 0$,} \\
                0, & \mbox{при $x < 0$.}
        \end{cases}
$$
        Будем обозначать такие случайные величины
$$
        X \sim \mbox{Exp}(\lambda).
$$
\end{definition}

Для того чтобы построить датчик экспоненциально распределенной с параметром $\lambda$ случайной величины $X$, воспользуемся доказанной ранее теоремой~\ref{th:th2-1}. Получается, что такую случайную величину можно представить в виде:
$$
        X =
        F_x^{-1}(\xi) =
        -\frac{1}{\lambda}\ln(1 - \xi),
$$
где $\xi$~--- равномерно распределенная на отрезке $[0,\,1]$ случайная величина.

\begin{assertion}[Свойство отсутствия памяти]
        Пусть $X\sim\mbox{Exp\,}(\lambda)$, тогда для любых $t \neq 0$ и $s$ справедливо:
$$
        \p(X\geqslant s+t\,|\,X\geqslant t) =
        \p(X \geqslant s).
$$
\end{assertion}

\begin{proof}
        Рассмотрим левую часть равенства:
$$
        \p(X \geqslant s + t\,|\,X\geqslant t) =
        \frac{\p(X \geqslant s + t,\,X\geqslant t)}{\p{X\geqslant t}} =
        \frac{\p(X \geqslant s + t)}{\p(X\geqslant t)}.
$$
Таким образом получаем, утверждение эквивалентно тому, что
$$
        \p(X\geqslant s+t) = 
        \p(X\geqslant t)\p(X\geqslant s).
$$
Из определения функции распределения $F_X(t) = \p(X < t) = 1 - \p(X \geqslant t)$ получаем, что
$$
        e^{-\lambda(s+t)} = e^{-\lambda s}e^{-\lambda t}.
$$
Последнее равенство точно верно. Таким образом, утверждение доказано.
\end{proof}