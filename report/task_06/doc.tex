\section{Задание №6}
\begin{enumerate}
        \item Посчитать интеграл
$$
        \int\limits_{-\infty}^{+\infty}
        \int\limits_{-\infty}^{+\infty}
        \ldots
        \int\limits_{-\infty}^{+\infty}
        \frac
        {
                e^{-\left(x_1^2 + \ldots +x_{10}^2 + \frac{1}{2^7 \cdot x_1^2 \cdot \ldots \cdot x_{10}^2}\right)}
        }
        {
                x_1^2 \cdot \ldots \cdot x_{10}^2
        }
        \, dx_1 dx_2 \ldots dx_{10}
$$
        \begin{itemize}
                \item[-] методом Монте--Карло;
                \item[-] методом квадратур, сводя задачу к вычислению собственного интеграла Римана.
        \end{itemize}
        \item Для каждого случая оценить точность вычислений.
\end{enumerate}


\subsection{Метод Монте--Карло}
В данном разделе найдем методом Монте--Карло искомый интеграл и оценим точность вычислений. Для начала обозначим данный интеграл и подынтегральную функцию
$$
        I
        =
        \int\limits_{-\infty}^{+\infty}
        \int\limits_{-\infty}^{+\infty}
        \ldots
        \int\limits_{-\infty}^{+\infty}
        \underbrace{
                \frac
                {
                        e^{-\left(x_1^2 + \ldots +x_{10}^2 + \frac{1}{2^7 \cdot x_1^2 \cdot \ldots \cdot x_{10}^2}\right)}
                }
                {
                        x_1^2 \cdot \ldots \cdot x_{10}^2
                }
        }_{f(x)}
        \, dx_1 dx_2 \ldots dx_{10}.
$$

Пусть $\rho(x)$ --- плотность некоторой случайной величины $X$, имеющей абсолютно непрерывное распределение. Обозначим за $\hat f(x) = \frac{f(x)}{\rho(x)}$. Тогда
$$
        \int\limits_{\R^n} f(x)\,dx
        =
        \int\limits_{\R^n} \frac{f(x)}{\rho(x)} \rho(x) \,dx
        =
        \int\limits_{\R^n} \hat f(x) \rho(x) \,dx
        =
        \E \hat f(X).
$$
Получается, что задача вычисления интеграла сводится к вычислению математического ожидания случайной величины $Y = \hat f(X)$. Согласно закону больших чисел (см. теорему~\ref{th:big_numbers})
$$
        Y_n
        =
        \frac {
                \sum_{i=1}^n \hat f(x^i)
        } {
                n
        }
        \xrightarrow{n \to \infty}
        \E \hat f(X),
        \quad
        \mbox{где $\{x^i\}_{i = 1}^n$ --- выборка случайной величины $X$.}
$$
Поэтому в дальнейшем будем считать математическое ожидание, полагая
$$
        \E \hat f(X) \approx Y_n.
$$

Для подсчета интеграла $I$ возьмем случайную величину $X$, имеющую десятимерное стандартное нормальное распределение, то есть $X \sim \mbox{N} (O,\,I)$, где $O,\,I \in \R^{10 \times 10}$, $O$ --- нулевая матрица, $I$ --- единичная матрица. Плотность такой случайной величины выражается формулой
$$
        \rho(x)
        =
        \frac{1}{(2\pi)^5}
        e^{-\frac{x_1^2 + \ldots + x_{10}^2}{2}}.
$$
Тогда
$$
\boxed{
        \hat f(x) = \frac{f(x)}{\rho(x)}
        =
        (2\pi)^5
        \frac
        {
                e^{-\left(\frac{x_1^2 + \ldots + x_{10}^2}{2} + \frac{1}{2^7 \cdot x_1^2 \cdot \ldots \cdot x_{10}^2}\right)}
        }
        {x_1^2 \cdot \ldots \cdot x_{10}^2}.
}
$$
Таким образом, мы можем посчитать значение интеграла $I$ по формуле
$$
\boxed{
        I
        \approx
        \frac
        {\sum_{i=1}^{n} \hat f(x^i)}
        {n},
        \quad
        \mbox{где $x^i = (x_1^i,\,\ldots,\,x_{10}^i)$, $x_k^i\sim\mbox{N}(0,\,1)$, $k = \overline{0,\,10}$.}
}
$$

Теперь оценим точность вычисления интеграла методом Монте--Карло. Для этого воспользуемся центоральной предельной теоремой (см. теорему \ref{th:central_limit}):
\begin{multline*}
        \p(|Y_n - I| < \varepsilon)
        =
        \p\left( \left| \frac{S_n - nI}{n} \right| < \varepsilon \right)
        =
        \p\left( \left| \frac{S_n - nI}{\sqrt{n}\sigma_n} \right| < \frac{\varepsilon\sqrt{n}}{\sigma_n} \right)
        =\\=
        \p\left( \frac{S_n - nI}{\sqrt{n}\sigma_n}  < \frac{\varepsilon\sqrt{n}}{\sigma_n} \right)
        -
        \p\left( \frac{S_n - nI}{\sqrt{n}\sigma_n}  < -\frac{\varepsilon\sqrt{n}}{\sigma_n} \right)
        \approx
        F_N\left(\frac{\varepsilon\sqrt{n}}{\sigma_n}\right) - F_N\left(-\frac{\varepsilon\sqrt{n}}{\sigma_n}\right)
        =\\=
        2 F_N\left(\frac{\varepsilon\sqrt{n}}{\sigma_n}\right) - 1
        =
        1 - \alpha.
\end{multline*}
$$
        \Downarrow
$$
$$
        \varepsilon = K_{1 - \nicefrac{\alpha}{2}}\cdot \frac{\sigma_n}{\sqrt{n}},
$$
где $\sigma_n$ --- несмещенная выборочная дисперсия, $F_N$ --- функция стандартного нормального распределения, а $K_\alpha$ --- его квантиль порядка $\alpha$. Таким образом для уровня значимости $\alpha = 0,05$ погрешность будет равна $\varepsilon = 1,96\frac{\sigma_n}{\sqrt{n}}$.

\begin{table}[h]
\begin{center}
\begin{tabular}{|c|c|c|c|}
\hline
Число испытаний &
Результат  &
Погрешность &
Время работы, с
\\
\hline
$10^5$
&
$127,814$
&
$23,0369$
&
$0,016521$
\\
\hline
$10^6$
&
$124,315$
&
$2,27042$
&
$0,997902$
\\
\hline
$10^7$
&
$124,791$
&
$0,720674$
&
$9,60839$
\\
\hline
$10^8$
&
$124,735$
&
$0,227662$
&
$104,135$
\\
\hline
\end{tabular}
\end{center}
\caption{Результат подсчета интеграла $I$ методом Монте--Карло.}
\end{table}


\subsection{Метод квадратур}

В методе квадратур мы должны свести задачу к вочислению собственного интеграла Римана. Для этого сделаем замену переменных
$$
        x_i = \tg\left(\frac{\pi}{2}t_i\right), \quad 0 \leqslant t_i \leqslant 1, \quad i = \overline{1, 10}.
$$
Таким образом, исходный интеграл равен интегралу с ограниченой областью интегрирования:
$$
        I
        =
        \left(\frac{\pi}{2}\right)^{10}
        \int\limits_{0}^{1}\ldots\int\limits_{0}^{1}\,
        \frac
        {
                \exp\left\{
                -\left[
                        \sum_{k=1}^{10}\tg\left(\frac{\pi}{2}t_k\right)^2
                        +
                        \frac
                        {
                                1 
                        }
                        {
                                2^7
                                \cdot
                                \prod_{k=1}^{10}\tg\left(\frac{\pi}{2}t_k\right)^2
                        }
                \right]
                \right\}
        }
        {
                \prod_{k=1}^{10}\tg\left(\frac{\pi}{2}t_k\right)^2
                \cdot
                \prod_{k=1}^{10}\cos\left(\frac{\pi}{2}t_k\right)^2
        }
        \,dt_1\ldots dt_{10}.
$$

Теперь обозначим подынтегральную функцию последнего выражения за $F(t) = F(t_1,\,\ldots,\,t_{10})$ и рассмотрим равномерное разбиение откезка $[0,\,1]$ на $N$ частей:
$$
        0 = t^0 < t^1 < \ldots < t^{10} = 1,
        \quad
        t^k = \frac{k}{N},
        \quad
        k = \overline{0, N}
$$
если взять за $\xi_k = \frac{t^{k-1} + t^k}{2}$ середины отрезков $[t^{k-1}, t^k]$, $k=\overline{1,N}$, то приближенно значение интеграла можно посчитать по формуле
$$
        I
        \approx
        \sum_{k_1 = 1}^N
        \ldots
        \sum_{k_{10} = 1}^N
        F(\xi_{k_1},\,\ldots,\,\xi_{k_{10}})\Delta,
        \quad
        \mbox{где } \Delta = \frac{1}{N^{10}}.
$$

Погрешность метода прямоугольников имеет следующий вид:
$$
        \varepsilon = \frac{h^2}{24}(b- a)\sum_{i,j=1}^{n}\max|f''_{x_ix_j}| = \frac{1}{6N^2}\sum_{i,j=1}^n\max|f''_{x_ix_j}|.
$$

\begin{table}[h]
\begin{center}
\begin{tabular}{|c|c|c|}
\hline
Число разбиений &
Результат  &
Время работы, с
\\
\hline
$3$
&
$0,086797$
&
$0,007513$
\\
\hline
$4$
&
$272,6029$
&
$0,073578$
\\
\hline
$5$
&
$183,4886$
&
$0,889859$
\\
\hline
$6$
&
$116,3903$
&
$8,377370$
\\
\hline
\end{tabular}
\end{center}
\caption{Результат подсчета интеграла $I$ методом квадратур. Видно, что метод Монте--Карло гораздо более оптимальный по времени.}
\end{table}
