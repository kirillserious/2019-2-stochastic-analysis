\section{Задание №6}
\begin{enumerate}
        \item Посчитать интеграл
$$
        \int\limits_{-\infty}^{+\infty}
        \int\limits_{-\infty}^{+\infty}
        \ldots
        \int\limits_{-\infty}^{+\infty}
        \frac
        {
                e^{-\left(x_1^2 + \ldots +x_{10}^2 + \frac{1}{2^7 \cdot x_1^2 \cdot \ldots \cdot x_{10}^2}\right)}
        }
        {
                x_1^2 \cdot \ldots \cdot x_{10}^2
        }
        \, dx_1 dx_2 \ldots dx_{10}
$$
        \begin{itemize}
                \item[-] методом Монте--Карло;
                \item[-] методом квадратур, сводя задачу к вычислению собственного интеграла Римана.
        \end{itemize}
        \item Для каждого случая оценить точность вычислений.
\end{enumerate}


\subsection{Метод Монте--Карло}
В данном разделе найдем методом Монте--Карло искомый интеграл и оценим точность вычислений. Для начала обозначим данный интеграл и подынтегральную функцию
$$
        I
        =
        \int\limits_{-\infty}^{+\infty}
        \int\limits_{-\infty}^{+\infty}
        \ldots
        \int\limits_{-\infty}^{+\infty}
        \underbrace{
                \frac
                {
                        e^{-\left(x_1^2 + \ldots +x_{10}^2 + \frac{1}{2^7 \cdot x_1^2 \cdot \ldots \cdot x_{10}^2}\right)}
                }
                {
                        x_1^2 \cdot \ldots \cdot x_{10}^2
                }
        }_{f(x)}
        \, dx_1 dx_2 \ldots dx_{10}.
$$

Пусть $\rho(x)$ --- плотность некоторой случайной величины $X$, имеющей абсолютно непрерывное распределение. Обозначим за $\hat f(x) = \frac{f(x)}{\rho(x)}$. Тогда
$$
        \int\limits_{\R^n} f(x)\,dx
        =
        \int\limits_{\R^n} \frac{f(x)}{\rho(x)} \rho(x) \,dx
        =
        \int\limits_{\R^n} \hat f(x) \rho(x) \,dx
        =
        \E \hat f(X).
$$
Получается, что задача вычисления интеграла сводится к вычислению математического ожидания случайной величины $Y = \hat f(X)$. Согласно закону больших чисел (см. теорему~\ref{th:big_numbers})
$$
        Y_n
        =
        \frac {
                \sum_{i=1}^n \hat f(x^i)
        } {
                n
        }
        \xrightarrow{n \to \infty}
        \E \hat f(X),
        \quad
        \mbox{где $\{x^i\}_{i = 1}^n$ --- выборка случайной величины $X$.}
$$
Поэтому в дальнейшем будем считать математическое ожидание, полагая
$$
        \E \hat f(X) \approx Y_n.
$$

Для подсчета интеграла $I$ возьмем случайную величину $X$, имеющую десятимерное стандартное нормальное распределение, то есть $X \sim \mbox{N} (O,\,I)$, где $O,\,I \in \R^{10 \times 10}$, $O$ --- нулевая матрица, $I$ --- единичная матрица. Плотность такой случайной величины выражается формулой
$$
        \rho(x)
        =
        \frac{1}{(2\pi)^5}
        e^{-\frac{x_1^2 + \ldots + x_{10}^2}{2}}.
$$
Тогда
$$
\boxed{
        \hat f(x) = \frac{f(x)}{\rho(x)}
        =
        (2\pi)^5
        \frac
        {
                e^{-\left(\frac{x_1^2 + \ldots + x_{10}^2}{2} + \frac{1}{2^7 \cdot x_1^2 \cdot \ldots \cdot x_{10}^2}\right)}
        }
        {x_1^2 \cdot \ldots \cdot x_{10}^2}.
}
$$
Таким образом, мы можем посчитать значение интеграла $I$ по формуле
$$
\boxed{
        I
        \approx
        \frac
        {\sum_{i=1}^{n} \hat f(x^i)}
        {n},
        \quad
        \mbox{где $x^i = (x_1^i,\,\ldots,\,x_{10}^i)$, $x_k^i\sim\mbox{N}(0,\,1)$, $k = \overline{0,\,10}$.}
}
$$

Теперь оценим точность вычисления интеграла методом Монте--Карло. Для этого воспользуемся центоральной предельной теоремой (см. теорему \ref{th:central_limit}):
\begin{multline*}
        \p(|Y_n - I| < \varepsilon)
        =
        \p\left( \left| \frac{S_n - nI}{n} \right| < \varepsilon \right)
        =
        \p\left( \left| \frac{S_n - nI}{\sqrt{n}\sigma_n} \right| < \frac{\varepsilon\sqrt{n}}{\sigma_n} \right)
        =\\=
        \p\left( \frac{S_n - nI}{\sqrt{n}\sigma_n}  < \frac{\varepsilon\sqrt{n}}{\sigma_n} \right)
        -
        \p\left( \frac{S_n - nI}{\sqrt{n}\sigma_n}  < -\frac{\varepsilon\sqrt{n}}{\sigma_n} \right)
        \\
\end{multline*}
